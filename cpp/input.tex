Inläsning och utskrift av data till terminalen gör man med hjälp av \texttt{cin} och \texttt{cout} som vi tidigare såg. Syntaxen är väldigt enkel:

\lstinputlisting{cpp/input.cpp}

Skriv av koden, kompilera den och kör. Vad gör programmet?

\texttt{cin} och \texttt{cout} kan användas med alla inbyggda numeriska typer, dvs \texttt{int}, \texttt{long long}, \texttt{double}, \texttt{char} osv, samt \texttt{string}.

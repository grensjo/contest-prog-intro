Än så länge har alla våra program följt en väldigt linjär struktur utan några hopp i källkoden. Det visar sig dock nödvändigt att kunna exekvera vissa satser beroende på om ett booleskt uttryck är sant eller inte.

Antag exempelvis att vi har ett tal $a$ och vill räkna ut absolutbeloppet av det. Absolutbeloppet av ett tal x defineras som bekant såhär:

$$
|x| = 
\left\{
    \begin{array}{ll}
        x & \mbox{om } x \geq 0 \\
        -x & \mbox{annars}
    \end{array}
\right.
$$

Lyckligtvis är detta en barnlek att göra i C++. Ordet ``om'' i den matematiska definitionen ersätts av nyckelordet ``if'' i C++, och nyckelordet ``else'' fungerar som ``annars''.

\lstinputlisting{cpp/villkor.cpp}

Men tänk om vi istället har en mer komplex funktion, som:

$$
f(x) = 
\left\{
    \begin{array}{lll}
        1 & \mbox{om } x = 1\\
        \frac{x}{2} & \mbox{om } x \mbox{ jämnt}\\
        3x+1 & \mbox{annars}
    \end{array}
\right.
$$

Vi kan då använda en "else-if-sats". Om den första if-satsen inte var sann, kommer vi att kolla om varje else-if-sats i ordning är sann eller inte (om sådan finns). Annars kommer else-satsen exekveras (återigen, om sådan finns). Exempelvis skulle koden kunna se ut på följande vis:


\lstinputlisting{cpp/villkor2.cpp}

Viktigt att notera är att om villkoret i en if-sats är sant så ignoreras samtliga efterföljande else-satser som hör till samma "else-if-kedja". Här är ett enkelt exempel:

\lstinputlisting{cpp/villkor3.cpp}

Innan vi går vidare tittar vi på ett sista exempel med villkor.

\lstinputlisting{cpp/villkor4.cpp}

